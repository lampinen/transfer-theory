\documentclass{beamer}
\usepackage{pgfpages}
%\setbeameroption{show notes on second screen=left} %enable for notes
\usepackage{graphicx}
\usepackage{xcolor}
\usepackage{listings}
%\usepackage{transparent}
\usepackage{hyperref}
\lstset{language=python,frame=single}
\usepackage{verbatim}
%\usepackage{apacite}
\usepackage[longnamesfirst]{natbib}
\usepackage{subcaption}
\usepackage{amsmath}
\usepackage{relsize}
\usepackage{appendixnumberbeamer}
\usepackage{xparse}
\usepackage{multimedia}
\usepackage{xcolor}
\usepackage[normalem]{ulem}
\usepackage{tikz}
\usetikzlibrary{matrix,backgrounds}
\usetikzlibrary{positioning}
\usetikzlibrary{shapes,arrows}
\usetikzlibrary{positioning}

\tikzset{onslide/.code args={<#1>#2}{%
  \only<#1>{\pgfkeysalso{#2}} 
}}

\tikzstyle{block} = [rectangle, draw, fill=red!20!blue!10, 
    text width=5em, text centered, rounded corners, minimum height=4em]
\tikzstyle{netnode} = [circle, draw, very thick, inner sep=0pt, minimum size=0.5cm] 
\tikzstyle{relunode} = [rectangle, draw, very thick, inner sep=0pt, minimum size=0.5cm] 
    
\tikzstyle{line} = [draw, line width=1.5pt, -latex']

\pgfdeclarelayer{background}
\pgfsetlayers{background,main}

\pgfdeclarelayer{myback}
\pgfsetlayers{myback,background,main}

\usetheme[numbering=fraction]{metropolis}
\newcommand{\semitransp}[2][35]{\color{fg!#1}#2}

\newcommand\blfootnote[1]{%
  \begingroup
  \renewcommand\thefootnote{}\footnote{#1}%
  \addtocounter{footnote}{-1}%
  \endgroup
}
\renewcommand*\footnoterule{}
%%\AtBeginSection[]
%%{
%%  \begin{frame}
%%    \frametitle{Table of Contents}
%%    \tableofcontents[currentsection]
%%  \end{frame}
%%}

\begin{document}

\title{Understanding generalization and transfer in deep linear neural networks}
\author{Andrew Lampinen}
\date{FriSem, 1/18/2019}
\frame{\titlepage}

\begin{frame}{Generalization in deep networks: AlphaGo's ``creativity''}
\begin{figure}
\includegraphics[width=0.5\textwidth]{figures/alphago_move_37.png}
\end{figure}
\note{Commentators called this move ``probably a mistake'' during the game, but later decided it was ``beautiful'' and unlike what humans had played before.}
\end{frame}

\begin{frame}{Generalization in humans: Past tense over-regularization}
\begin{figure}
\includegraphics[width=0.8\textwidth]{figures/past_tense.png}
\end{figure}
\note{There are many interesting phenomena like this, but there's a general pattern of learning broad features followed by progressive differentiation of lower level structure.}
\end{frame}

\begin{frame}[standout]
How, why, and \emph{when} do neural networks (and humans) generalize?
\end{frame}



\end{document}


